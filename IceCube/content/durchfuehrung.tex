\section{Durchführung}
Zunächst wird der Datensatz wie in Kapitel \ref{Signal} aufbereitet. Danach lässt sich mit Hilfe der mRMR-Selektion eine Attributsauswahl treffen. Mit den ausgewählten Attributen wird anschließend eine Seperation mit verschiedenen Lernern durchgeführt. Dabei erfolgt die Seperation mit einem Naiven-Bayes-Klassifikator, einen kNN-Klassifikator und einem Random-Forest. \\
Um einen Vergleich der verschiedenen Lerner zu ermöglichen, werden zuletzt die Qualitätsparameter für alle drei Methoden bestimmt und verglichen.
