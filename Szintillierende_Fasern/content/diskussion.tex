\section{Diskussion}
Im Laufe der Versuchsauswertung wurden die Reflektionskoeffizienten als Funktion des Winkels und der Wellenlänge bestimmt, sowie einige Eigenschaften der szintillierenden Fasern untersucht.\\

Die Auswertung basiert auf der Annahme einer radialsymmetrischen Verteilung, welche durch eine Messung in Abbildung \ref{fig:radial_test} getestet wurde. Anhand dieser Messung lässt sich die Annahme allerdings nicht eindeutig verifizieren, da zu wenig Datenpunkte im negativen Winkelbereich aufgenommen wurden. Zur eindeutigen Bestätigung der Annahme sollte die Messung mit mehr Datenpunkten, gleichmäßig um die Null verteilt, überprüft werden.\\

Die Überprüfung der Abdunklung in Abbildung \ref{fig:Untergrund_Untersuchung} weist für einige Winkelbereiche auf Lücken in der Abdeckung hin. Durch die zusätzliche Aufnahme und Subtraktion des Dunkelstromes von den Messwerten, sollte dieses allerdings behoben sein und keine Fehlerquelle mehr darstellen.\\

Die Aufnahme der Messwerte, sowie das Positionieren der LED und des Spektrometers erfolgte automatisch mit Hilfe eines Computerprogrammes. Somit sind Fehler in der Datenaufnahme unwahrscheinlich.\\

Der Fit der Abschwächung der Intensität als Funktion der Anregungeposition in Abbildung \ref{fig:a_I0_winkel} weist besonders für die Anfangsintensität $I_0$ lediglich kleine Fehler auf. Im Vergleich mit dem identischen Fit der Simulationsdaten in Abbildung \ref{fig:fit_exp_simu} ist zu beachten, dass die Simulationsdaten durch das Eingrenzen der Daten auf Kernphotonen in einem deutlich kleineren Winkelbereich gefittet wurden. Innerhalb dieses Bereiches stimmen die Verläufe der Fitparameter in etwa überein.\\
Es fällt weiterhin der starke Abfall der Messdaten im Bereich großer Winkel ins Auge. Anhand der Auswertung der Simulationsdaten kann an dieser Stelle davon ausgegangen werden, dass in diesem Bereich die Intensität der Mantelphotonen zunimmt. Da diese nicht durch die Fitformel beschrieben werden, ergeben sich in diesen Bereichen unphysikalische Ergebnisse.\\
Um die Verläufe besser zu vergleichen, können mehr Datenpunkte im relevanten Bereich kleiner Winkel aufgenommen werden.\\

Dieses Problem wird besonders deutlich, wenn die Messdaten auf den Winkelbereich der Kernphotonen zugeschnitten werden. Dort liegen jeweils lediglich drei Datenpunkte, welche sehr ungenau zu fitten sind, wie im Fit der winkelabhängigen Abschwächung in Abbildung \ref{fig:a0_epsilon} zu erkennen ist. Da an dieser Stelle nicht von einem physikalisch sinnvollen Verlauf der erhaltenen wellenlängenabhängigen Fitparameter ausgegangen werden kann, sollten mehr Daten in diesem Bereich aufgenommen werden. Der erwartete exponentielle Abfall des Reflektionskoeffizienten $\epsilon$ mit der Wellenlänge ist in Abbildung \ref{fig:a0_epsilon} nicht auszumachen.\\

Dasselbe gilt für den Reflektionskoeffizienten als Funktion des Winkels in Abbildung \ref{fig:reflektionskoeff_winkel}. Anhand der vier Werte, welche allesamt hohe Unsicherheiten besitzen, lässt sich keine Aussage über den physikalischen Sinn des Ergebnisses treffen.

