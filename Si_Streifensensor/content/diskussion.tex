\section{Diskussion}

Die in Abschnitt \ref{kap:Kennlinie} erstellte Kennlinie zeigt in beiden Fällen den erwarteten Verlauf. Die Leckstromwerte der zweiten Messung sind hierbei im Vergleich zur ersten Messung deutlich erhöht. Die zweite Messreihe wurde im Anschluss an die Messungen mit dem Laser aufgenommen. Im Zuge der Bestrahlung mit dem Laser kann sich das Detektormodul erwärmt haben, wodurch die Anzahl der freien Ladungsträger durch thermische Anregung erhöht ist. Hiermit lassen sich die erhöhten Stromwerte erklären.\\

Beide Kennlinien gehen im selben Bereich in einen Sättigungsverlauf über. Die geschätzte Depletionsspannung von $ U_{\mathrm{Dep}} \approx \SI{61}{\volt}$ wird durch die mittels Laser bestimmte Kennlinie in Abschnitt \ref{kap:CCEQ} bestätigt. Die Messungen mittels des Lasers ergeben einen höheren Wert von $ U_{\mathrm{Dep}} \approx \SI{100}{\volt}$. Der Hersteller \textit{Alibavasystems} gibt als Intervall für die Depletionsspannung $\SI{60}{\volt} < U < \SI{80}{\volt}$ an \cite{alibava}. Der im ersten Versuchsteil abgeschätzte Wert liegt innerhalb dieses Intervalles, während der Wert mittels Lasermessung die obere Intervallgrenze deutlich übersteigt.\\

Bei der Untersuchung des Untergrundes in Abschnitt \ref{kap:Pedestal} fällt die Verschiebung der gefundenen Verteilung des Common-Mode-Shifts im Vergleich zur erwarteten Gaußverteilung ins Auge. Ein Fehler in der Aufnahme der ADCC ist an dieser Stelle, wie auch in den weiteren Versuchsteilen auszuschließen, da die Datenaufnahme elektronisch erfolgte.
Vermuten lassen sich an dieser Stelle statistische Schwankungen der Werte. Um diese zu bestätigen oder um die Gaußverteilung der Common-Mode-Shift-Werte um null zu wiederlegen, sollte eine Messung mit einer höheren Eventanzahl durchgeführt werden.\\

Die in Abschnitt \ref{kap:Vermessung} ermittelte \textit{pitch} von $b = \SI{155(25)}{\micro\metre}$ weist eine Abweichung von $\SI{3.125}{\%}$ zur Herstellerangabe von $b_{\mathrm{exakt}} = \SI{160}{\micro\metre}$ auf \cite{alibava} und stimmt somit relativ gut mit diesem überein. Eine Fehlerquelle kann in der unzureichenden Fokussierung des Lasers gefunden werden. Für die Ausdehnung des Lasers auf dem Sensor wurde ein Wert von $d = \SI{292(17)}{\micro\metre}$ ermittelt, welcher für einen fokussierten Laser relativ hoch erscheint. Aus Abbildung \ref{fig:Position} wird ersichtlich, dass die Laserausdehnung die Breite eines Streifens übersteigt. Dadurch sinkt die Ortsauflösung der Laservermessung und die Positionsbestimmungen werden ungenau.\\

Auch die Berechnung der Eindringtiefe von $a= \SI{179(7)}{\micro\metre}$ über die Länge der an- und absteigenden Flanken wird durch die Ungenauigkeit der Positionsbestimmung beeinflußt.
Aus Abbildung \ref{fig:KennlinieLaser} ist außerdem ersichtlich, dass der Anstieg der CCE relativ zügig erfolgt und daher nur wenige Messwerte durch die Regression beschrieben werden. Für eine exaktere Bestimmung der Eindringtiefe sollten daher mehr Werte unterhalb der Depletionsspannung aufgenommen werden.
Für eine fehlerfreie Vermessung der Sensorstreifen und Lasereigenschaften sollten die Messungen des Weiteren mit einer sorgfältigeren Fokussierung des Lasers wiederholt werden.\\

Einen Einfluss der Laserfokussierung auf die CCE ist nicht zu erwarten, sodass diese Fehlerquelle im Vergleich mit der CCE der $^{90}$Sr-Quelle nicht berücksichtigt werden muss.
Der Energieverlust von Elektronen im Sensor wird durch die modifizierte Bethe-Bloch-Gleichung \ref{eq:12} beschrieben. Ihre Energiedeposition im Detektor ist ein statistischer Prozess. Die Laserphotonen hingegen deponieren im Falle eines vollständig depletierten Streifen ihre gesamte Energie im Detektor, welches die berechnete Eindringtiefe von $a= \SI{179(7)}{\micro\metre}$ bei einer Sensordicke von $D= \SI{300}{\micro\metre}$ bestätigt. Die Energiedeposition der Laserphotonen ist somit lediglich von der Breite der Depletionszone und somit der Vorspannung abhängig, während die Energiedeposition der Elektronen zusätzlich einer statistischen Verteilung unterliegt.

Die in Abschnitt \ref{kap:Quelle} erstellten Energiespektren zeigen den erwarteten Verlauf der Faltung einer Gauß- mit einer Landauverteilung. Laut der Versuchsanleitung \cite{anleitung} beträgt der durchschnittliche Energieverlust von Elektronen in Silizium $\frac{\mathrm{d}E}{\mathrm{d}x} = \SI{3.88}{\mega\electronvolt\per\centi \per \metre}$, was einem durchschnittlichen Energieverlust von $\frac{\mathrm{d}E}{\mathrm{d}x} = \SI{116.4}{\kilo\electronvolt\per 300\; \micro \metre}$ im Streifensensor entspricht. Der bestimmte Mittelwert des Energiespektrums von $\bar{E} = \SI{103.53 \pm 0.06}{\kilo\electronvolt}$ weist eine Abweichung von $\SI{11.06}{\%}$ zu diesem auf. Ein Grund für die Abweichung ist in der Tatsache zu finden, dass durch das Verwerfen von Energiedepositionen oberhalb von $\SI{250}{ADCC}$ aufgrund der begrenzten Umrechnungsfunktion das Spektrum eine leichte Verschiebung zu kleineren Energiewerten erfahren.
