\section{Diskussion}
Ab einer Signifikanz von $5\;\sigma$ wird in der Teilchenphysik von einer Entdeckung gesprochen. Der ermittelte Wert von $ S_{CP} = 7.89\;\sigma$ liegt deutlich über diesem Wert. Somit kann anhand dieser Analyse von einer Existenz der $CP$-Verletzung im Zerfall von $B$-Mesonen in drei Kaonen gesprochen werden.

Diese wird ebenfalls bei einem Blick in Abbildung \ref{fig:AsymmMassen} deutlich. Bei der Kollision von zwei Protonen sind in dem betrachteten Phasenraumbereich deutlich mehr $B^{+}$- als $B^{-}$-Mesonen erzeugt worden.

Die Angabe der systematischen Unsicherheit wurde in der Analyse lediglich auf die Produktionsasymmetrie beschränkt. Weitere systematische Unsicherheiten können beispielsweise durch verschiedene Detektionseffizienzen in unterschiedlichen Detektorbereichen, sowie von Teilchen und Antiteilchen zu finden sein. Eine genauere Angabe der Unsicherheiten kann einen beachtlichen Einfluss auf die Signifikanzwerte haben.

Die Bereiche, in denen signifikante Anzeichen für eine $CP$-Verletzung zu
erkennen sind, variieren stark mit den gewählten Grenzen zur Datenselektion.
Anhand der Abbildungen \ref{fig:probK} und \ref{fig:probPi} ist zu vermuten,
dass durch die gewählten Grenzen eine beträchtliche Anzahl an Ereignissen mit
Kaonen verworfen werden und eine hohe Anzahl an Untergrundereignissen in den
Daten verbleiben. Im Laufe der Versuchsdurchführung sind aus diesem Grund
engere Selektionen ausprobiert worden. In diesem Fall sind kleinere
Signifikanzwerte erreicht worden.
Für eine Verifizierung der Analyseergebnisse sollte aus diesem Grund über
Methoden einer eindeutigeren Teilchenidentifizierung nachgedacht werden.

Des Weiteren wurden im Rahmen dieses Versuches lediglich Zerfälle in drei
Kaonen betrachtet. Der $B$-Mesonenzerfall erfolgt allerdings auch über andere
Zerfallskanäle, wie zum Beispiel in drei Pionen oder Mischzustände von Pionen
und Kaonen. Eine Untersuchung dieser Zerfälle auf $CP$-Asymmetrie könnte
interessante Einsichten liefern.
