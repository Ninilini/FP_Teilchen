\section{Durchführung}
Zur Bestimmung des Fluss des Krebsnebels werden die Daten zunächst selektiert. Dazu wird die zuvor bestimmte Gamma-Hadron-Seperation von $0.8$ angewendet. Dazu werden nur die Daten behalten, die eine \textit{gamma\_prediction} \textgreater \; $0.8$ haben. Zudem muss der Abstand zur Quellposition \textit{theta\_deg} bzw. zur $i$-te Off-Position einen Wert von $\sqrt{0.025}\si{\degree}$ haben.\\
Die für die Entfaltung notwendige Energie-Migrationsmatrix lässt sich bestimmen, indem die \textit{gamma\_energy\_prediction}, die geschätzte Energie des Random Forest, gegen die \textit{corsika\_event\_header\_total\_energy}, die wahre Energie der simulierten Gammas, aufgetragen wird.\\
Im Anschluss lässt sich eine Entfaltung mit der naiven SVD-Methode und einer Poisson-Likelihood durchführen.\\
Zur korrekten Berechnung des Flusses muss jedoch noch die Akzeptanzkorrektur durchgeführt werden. Sie wird als effektive Fläche $A_{eff}$ eines perfekten Detektors angegeben. Dabei wird die tatsächliche abgedeckte Detektorfläche $A$ mit der Wahrscheinlichkeit $p(E)$ ein Ereignis der wahren Energie zu messen multipliziert. Dadurch ergibt sich für die Akzeptanzkorrektur:\\
\begin{align}
	A_{eff,i} &= \frac{N_{selektiert,i}}{N_{simuliert,i}} \cdot A \; \text{mit}\\
	A &= \pi \cdot (\SI{270}{\meter})^2 = \SI{229022.1044}{\meter}^2
\end{align}
Zusammen mit den entfalteten Ereigniszahlen $\hat{\vec{f}}$ berechnet sich der Fluss des Krebsnebels zu:
\begin{align}
	\Phi_{i} = \frac{\hat{\vec{f}}}{A_{eff,i} \cdot \Delta E_{i} \cdot t_{obs}}
\end{align}
Dabei ist $\Delta E_{i}$ die Breite des Energiebins und $t_{obs}$ die Beobachtungsdauer, welche aus der Spalte \textit{ontime} der \textit{runs} Tabelle entnommen werden kann.
