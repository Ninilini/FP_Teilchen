\section{Durchführung}
Zur Bestimmung des Fluss des Krebsnebels werden die Daten zunächst selektiert. Dazu wird die zuvor bestimmte Gamma-Hadron-Separation von $0.8$ angewendet. Dazu werden nur die Daten behalten, die eine \textit{gamma\_prediction} \textgreater \; $0.8$ haben. Zudem muss der Abstand zur Quellposition \textit{theta\_deg} bzw. zur $i$-te Off-Position einen Wert von maximal $\sqrt{0.025}\si{\degree}$ haben.\\
Die für die Entfaltung notwendige Energie-Migrationsmatrix lässt sich bestimmen, indem die \textit{gamma\_energy\_prediction}, die geschätzte Energie des Random Forest, gegen die \textit{corsika\_event\_header\_total\_energy}, die wahre Energie der simulierten Gammas, in einem zweidimensionalen Histogramm aufgetragen wird. Hierbei muss auf eine geeignete Normierung geachtet werden.\\
Im Anschluss lässt sich eine Entfaltung mit der naiven SVD-Methode und einer Poisson-Likelihood durchführen.\\
Zur korrekten Berechnung des Flusses muss jedoch noch die Akzeptanzkorrektur durchgeführt werden. Sie wird als effektive Fläche $A_{\text{eff}}$ eines perfekten Detektors angegeben. Dabei wird die tatsächliche abgedeckte Detektorfläche $A$ mit der Wahrscheinlichkeit $p(E)$ ein Ereignis der wahren Energie zu messen multipliziert. Dadurch ergibt sich für die Akzeptanzkorrektur:\\
\begin{align}
	A_{\text{eff},i} &= \frac{N_{\text{selektiert},i}}{N_{\text{simuliert},i}} \cdot A \; \text{mit}\\
	A &= \pi \cdot (\SI{27000}{\centi\meter})^2 = \SI{229022.1}{\centi\meter}^2
	\label{eqn:Aeff}
\end{align}
Zusammen mit den entfalteten Ereigniszahlen $\hat{{\pmb{f}}}$ berechnet sich der Fluss des Krebsnebels zu:
\begin{align}
	\Phi_{i} = \frac{\hat{f_{\mathrm{i}}}}{A_{\text{eff},i} \cdot \Delta E_{i} \cdot t_{\text{obs}} }
	\label{eqn:phi}
\end{align}
Dabei ist $\Delta E_{i}$ die Breite des Energiebins und $t_{\text{obs}}$ die Beobachtungsdauer, welche aus der Spalte \textit{ontime} der \textit{runs} Tabelle entnommen werden kann.
