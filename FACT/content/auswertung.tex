\section{Auswertung}
Zur Auswertung wird die Programmiersprache \texttt{python (Version 3.7.2)} mit
den Bibliothekserweiterungen \texttt{numpy}~\cite{numpy} und \texttt{matplotlib}~\cite{matplotlib} verwendet.\\
Zu Beginn wird der Cut auf die \textit{gamma\_prediction} angewendet. Im Anschluss wird sich das Eregbnis der Analyse \cite{FACTanalyse}, dass zu den vorleigenden Datensätzen geführt hat, noch einmal vor Augen geführt. Dazu werden in dem in Abbildung \ref{fig:Theta2} aufgeführten Theta-Quadrat-Plot der Abstand von der rekonstruierten Quellposition zur angenommenen Quellposition aufgenommen. \\
\begin{figure}
  \centering
  \includegraphics[width=0.8\textwidth]{plots/On_Off.pdf}
  \caption{Der Plot zeigt den Abstand zwischen der rekonstruierten Quellposition und der angenommenen Quellposition. Dabei wird in dunkelblau die Position des Krebsnebels(ON) dargestellt und in grün die fünf Untergrundregionen(Off).}
  \label{fig:Theta2}
\end{figure}
Aus dem Theta-Quadrat-Plot lässt sich ablesen, dass $\theta = \sqrt{0.025}\si{\degree}$ eine gute Seperation zwischen Signal und Untergrund ermöglicht.\\
Mithilfe der \texttt{2dhistogramm}-Funktion von \texttt{numpy} lässt sich die Energie-Migrationsmatrix bestimmen. Diese ist in Abbildung \ref{fig:mig} dargestellt.\\
\begin{figure}
  \centering
  \includegraphics[width=0.4\textwidth]{plots/Matrix.pdf}
  \caption{Der Plot zeigt die Energie-Migrationsmatrix, die zwischen den gemessenen Energieveteilungen und den physikalischen Energieverteilungen vermittelt.}
  \label{fig:mig}
\end{figure}
Um nach Formel \eqref{eqn:NSVD} eine naive SVD-Entfaltung durch zuführen müssen die gemessenen Ereignisse des Krebsnebels und der Untergrund gebinnt werden. Das Binning wird analog zu dem der Migrationsmatrix gewählt. Die beiden erhaltenden Verteilungen werden logarithmisch aufgetragen und sind in Abbildung \ref{fig:UG_Sig} dargestellt. Dabei ist zubeachten, dass der Untergrund zusätzlich mit $a=0.2$ gewichtet wird, da der Untergrund aus fünf Off-Positionen stammt.\\
\begin{figure}
  \centering
  \includegraphics[width=0.8\textwidth]{plots/Energieverteilung_UG_SIG.pdf}
  \caption{Der Plot zeigt die gebinnte Energieverteilung des Untergrunds in grün und die gebinnte, gemessene Energieverteilung des Krebsnebels in dunkelblau.}
  \label{fig:UG_Sig}
\end{figure}
Da die Energie-Migrationsmatrix nicht symmetrisch ist, wird sie mit Hilfe einer Moore-Penrose-Pseudoinverse invertiert. Zusammen mit den gebinnten Energieverteilungen ergibt sich das entfaltete Energiespektrum des Krebnebels nach Formel \eqref{eqn:NSVD} zu:
\begin{align}
	\hat{\vec{f}} = \begin{pmatrix}
			282.908\\
			227.478\\
			89.667\\
			33.424\\
			-0.425\\
	\end{pmatrix}
\end{align}
\begin{figure}
  \centering
  \includegraphics[width=0.8\textwidth]{plots/Entfaltung_1.pdf}
  \caption{geeignete Beschreibung einfügen.}
  \label{fig:E1}
\end{figure}
