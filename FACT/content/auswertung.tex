\section{Auswertung}
Zur Auswertung wird die Programmiersprache \texttt{python (Version 3.7.2)} mit
den Bibliothekserweiterungen \texttt{numpy}~\cite{numpy}, \texttt{scipy}~\cite{scipy} und \texttt{matplotlib}~\cite{matplotlib} verwendet.\\
Zu Beginn wird der Cut auf die \textit{gamma\_prediction} angewendet. Im Anschluss wird sich das Eregbnis der Analyse \cite{FACTanalyse}, dass zu den vorleigenden Datensätzen geführt hat, noch einmal vor Augen geführt. Dazu werden in dem in Abbildung \ref{fig:Theta2} aufgeführten Theta-Quadrat-Plot der Abstand von der rekonstruierten Quellposition zur angenommenen Quellposition aufgenommen. \\
\begin{figure}
  \centering
  \includegraphics[width=0.65\textwidth]{plots/On_Off.pdf}
  \caption{Der Plot zeigt den Abstand zwischen der rekonstruierten Quellposition und der angenommenen Quellposition. Dabei wird in dunkelblau die Position des Krebsnebels(ON) dargestellt und in grün die fünf Untergrundregionen(Off).}
  \label{fig:Theta2}
\end{figure}
Aus dem Theta-Quadrat-Plot lässt sich ablesen, dass $\theta = \sqrt{0.025}\si{\degree}$ eine gute Seperation zwischen Signal und Untergrund ermöglicht.\\
Mithilfe der \texttt{2dhistogramm}-Funktion von \texttt{numpy} lässt sich die Energie-Migrationsmatrix bestimmen. Diese ist in Abbildung \ref{fig:mig} dargestellt.\\
\begin{figure}
  \centering
  \includegraphics[width=0.4\textwidth]{plots/Matrix.pdf}
  \caption{Der Plot zeigt die Energie-Migrationsmatrix, die zwischen den gemessenen Energieveteilungen und den physikalischen Energieverteilungen vermittelt.}
  \label{fig:mig}
\end{figure}
\subsection{Naive SVD-Entfaltung}
Um nach Formel \eqref{eqn:NSVD} eine naive SVD-Entfaltung durch zuführen müssen die gemessenen Ereignisse des Krebsnebels und der Untergrund gebinnt werden. Das Binning wird analog zu dem der Migrationsmatrix gewählt. Die beiden erhaltenden Verteilungen werden logarithmisch aufgetragen und sind in Abbildung \ref{fig:UG_Sig} dargestellt. Dabei ist zubeachten, dass der Untergrund zusätzlich mit $a=0.2$ gewichtet wird, da der Untergrund aus fünf Off-Positionen stammt.\\
\begin{figure}
  \centering
  \includegraphics[width=0.8\textwidth]{plots/Energieverteilung_UG_SIG.pdf}
  \caption{Der Plot zeigt die gebinnte Energieverteilung des Untergrunds in grün und die gebinnte, gemessene Energieverteilung des Krebsnebels in dunkelblau.}
  \label{fig:UG_Sig}
\end{figure}
Da die Energie-Migrationsmatrix nicht symmetrisch ist, wird sie mit Hilfe einer Moore-Penrose-Pseudoinverse invertiert. Zusammen mit den gebinnten Energieverteilungen ergibt sich das entfaltete Energiespektrum des Krebnebels nach Formel \eqref{eqn:NSVD} zu:
\begin{align*}
	\hat{\vec{f}_{\text{NSVD}}} = \begin{pmatrix}
			319.374\\
			196.335\\
			96.902\\
			12.454\\
			1.330\\
	\end{pmatrix}
\end{align*}
\begin{figure}
  \centering
  \includegraphics[width=0.8\textwidth]{plots/Entfaltung_1.pdf}
  \caption{Der Plot zeigt die entfaltete Energieverteilung des Krebsnebels. Zum Entfalten wurde wurde die naive SVD-Entfaltung genutzt.}
  \label{fig:E1}
\end{figure}
\subsection{Likelihood-Entfaltung}
Die Lösung der Likelihood-Entfaltung \eqref{eqn:fLike} kann nicht analytisch bestimmt. Daher wird mithilfe der Funktion \textit{scipy.optimize.minimize} die negative Log-Likelihood \eqref{eqn:loglike} numerisch minimiert. Damit ergibt sich der Schätzer $\hat{\vec{f}_{\text{Like}}}$ zu:\\
\begin{align*}
	\hat{\vec{f}_{\text{Like}}} = \begin{pmatrix}
			317.702\\
			199.230\\
			94.170\\
			14.211\\
			1.000\\
	\end{pmatrix}
\end{align*}
Die entfaltete Energieverteilung befindet sich in Abbildung \ref{fig:E2}.
\begin{figure}
  \centering
  \includegraphics[width=0.8\textwidth]{plots/Entfaltung_2.pdf}
  \caption{Der Plot zeigt die entfaltete Energieverteilung des Krebsnebels. Zum Entfalten wurde wurde die naive SVD-Entfaltung genutzt.}
  \label{fig:E2}
\end{figure}
\subsection{Der Fluss des Krebsnebels}
Um den Fluss des Krebsnebels nach Formel \eqref{eqn:phi} zu brechnen, müssen zunächst die Oberservationszeit $t_{\text{obs}}$, die Breite der Energiebins $\Delta E_{i}$ und nach Formel \eqref{eqn:Aeff} die Akzeptanzkorrektur berechnet werden.\\
Die Oberservationszeit beträgt $t_{\text{obs}} = \SI{63815.89}{\second}$. Die Breite der Energiebins ergibt sich zu:\\
\begin{align*}
	\Delta E_{i} = \begin{pmatrix}
			0.756\\
			1.899\\
			4.770\\
			11.981\\
			30.095\\
	\end{pmatrix}
\end{align*}
Die Akzeptanzkorrektur ergibt sich zu:
\begin{align*}
	A_{\text{eff},i} = \begin{pmatrix}
			3.607\cdot10^{7}\\
			2.409\cdot10^{8}\\
			3.652\cdot10^{8}\\
			3.760\cdot10^{8}\\
			1.748\cdot10^{8}\\
	\end{pmatrix}
\end{align*}
Damit ergibt sich der Fluss des Krebsnebels für die beiden Entfaltungsarten zu:
\begin{align*}
	\Phi_{\text{NSVD}} = \begin{pmatrix}
			1.835\cdot10^{-10}\\
			6.725\cdot10^{-12}\\
			8.716\cdot10^{-13}\\
			4.332\cdot10^{-14}\\
			3.959\cdot10^{-15}\\
	 \end{pmatrix} &\;\;\; \Phi_{\text{Like}}  = \begin{pmatrix}
				1.826\cdot10^{-10}\\
				6.824\cdot10^{-12}\\
				8.469\cdot10^{-13}\\
				4.866\cdot10^{-14}\\
				3.034\cdot10^{-15}\\
		\end{pmatrix}
\end{align*}
In Abbildung \ref{fig:Fluss} befindet sich eine graphische Darstellung des Flusses des Krebsnebels.\\
\begin{figure}
  \centering
  \includegraphics[width=0.8\textwidth]{plots/Fluss.pdf}
  \caption{Der Plot zeigt in grün den berechneten Fluss des Krebsnebels mit dem Schätzer der naiven SVD-Entfaltung. In blau ist der berechnete Fluss des Krebsnebels mit dem Schätzer der Likelihood-Entfaltung dargestellt.}
  \label{fig:Fluss}
\end{figure}
Für die Likelihood-Entfaltung lässt sich zudem eine Fehlerrechnung mithilfe der Kovarianzmatrix brechnen. Dabei ergibt sich der Fluss zu:
\begin{align*}
	\Phi_{\text{Like}} = \begin{pmatrix}
			1.825\cdot10^{-10}\pm 2.046\cdot10^{-11}\\
			6.824\cdot10^{-12}\pm 7.778\cdot10^{-13}\\
			8.469\cdot10^{-13}\pm 1.338\cdot10^{-13}\\
			4.866\cdot10^{-14}\pm 3.766\cdot10^{-14}\\
			3.034\cdot10^{-15}\pm 2.999\cdot10^{-15}\\
	 \end{pmatrix}
\end{align*}
In Abbildung \ref{fig:FlussLike} befindet sich das Ergebnis der Fluss-Berechnung mit Fehlerrechnung. Der $x$-Fehler ergibt sich dabei über die Binbreiten.
\begin{figure}
  \centering
  \includegraphics[width=0.8\textwidth]{plots/Fluss_Like.pdf}
  \caption{Der Plot zeigt den berechneten Fluss des Krebsnebels mit Fehlerrechnung. Der verwendete Schätzer wurde mit der Likelihood-Entfaltung bestimmt.}
  \label{fig:FlussLike}
\end{figure}
